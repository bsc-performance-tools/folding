
\chapter{API}

This section covers the public API available in the \FOLDING package.
This API is meant to allow the \FOLDING tool to interact with other performance analysis tools in addition to \EXTRAE.

\section{Generation of input files for the \FOLDING}

\subsection{Usage example}

The directory \texttt{\$FOLDING\_HOME/share/examples/folding-writer} contains an example that shows how to generate an input file for the folding from a programatically point of view.
The example can be compiled using the following command:

\begin{verbatim}
# cd $FOLDING_HOME/share/examples/folding-writer
# make
\end{verbatim}

The Listing~\ref{lst:ExampleGenerationInterpolation} shows the example provided in the distributed/installed package.
This example demonstrates how to programatically create an \texttt{.extract} file for the \texttt{interpolate} binary of the \FOLDING package.

\input{listings/user-guide/folding-writer.tex}

The given example considers that the region \texttt{FunctionA} has been identified somehow by the underlying monitoring mechanism, starts at 1,000~ns and lasts 4,500~ns (lines 31-33).
Within this period of time, three samples have occurred (\textit{s1-s3}, created in lines 40, 50 and 58, respectively).
Samples contain performance counter information and source code references.
The performance counter information is given in a relative manner, thus each sample contains the difference from the previous sample (or starting point).
For instance, sample \textit{s1} captured information from two performance counters (\texttt{PAPI\_TOT\_INS} and \texttt{PAPI\_TOT\_CYC}) that counted 1,000 and 2,000 events since the start of the region at time-stamp 2,000~ns (lines 36-40).
The second sample (\textit{s2}) does not only contain information from performance counters, but also contains a call-stack segment referencing two call-stack frames.
The first frame (\texttt{codeinfo\_l0}) refers to the routine coded as 1, which has source code information coded as 2, and AST-block information coded as 3 (line 46).
The same applies to second frame (\texttt{codeinfo\_l1}) - (line 47).
These frames are mapped into depths 0 and 1 (where 0 refers to the top of the call-stack) in lines 48 and 49, and then the sample is built using the performance counter information and the call-stack information in line 50.
Finally, the last sample (\textit{s3}) only accounted 500 and 1,000 events for the \texttt{PAPI\_TOT\_INS} and \texttt{PAPI\_TOT\_CYC} performance counters respectively, but did not capture any source code reference (lines 54-58).
This last sample should coincide with the end of the region (\texttt{FunctionA}), and may not be necessarily information captured from a sample point, but from an instrumentation point that indicates the end of the region.
All these samples are packed together in a STL vector container (lines 60-63), and then the \texttt{FoldingWriter::Write} static method dumps all the information from the samples using the given output stream (lines 65-71).

